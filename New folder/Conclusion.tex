\section{Conclusion and implications for management}

Differences in fish assemblage structure between Santa Cruz and Floreana are more likely the result of different water conditions rather than geographical distance or different physical habitats. However, while the fish assemblages in Santa Cruz were mainly affected by water quality, fish assemblages in Floreana were mainly affected by the type of physical habitats. This difference was assigned to the different levels of anthropogenic pressure on the coastal waters of both islands, with Floreana having less anthropogenic pressures and a more pristine nature than Santa Cruz. In Santa Cruz, higher levels of pollution seem to correspond with a lower $\alpha$ diversity and significantly different structure of fish assemblages, highlighting the importance of environmental and waste management in the populated bays of the Galapagos archipelago and confirming the negative effects of anthropogenic pressures on fish assemblages of tropical oceanic islands \citep{Quimbayo2019DeterminantsIslands,Sandin2008IslandFish}. 

The structure of fish communities can be used as an indicator of human pollution. However, care should be taken as anthropogenic pressures and natural stressors can easily be confounded. Therefore, when assessing coastal water quality and anthropogenic pressures, naturally occurring local and seasonal variations should be taken into account. Although the sampling effort was distributed in such a way to provide reliable estimates of the underlying environmental drivers of fish assemblage structure and diversity, separating anthropogenic from natural stressors remained difficult. In this study, only anthropogenically influenced areas, with distinct natural stressors, e.g. temperature, were studied. Future studies should include more islands and stronger gradients of anthropogenic pressures to confirm the found results. In addition, estimates of fishing pressures and boat traffic should be used in future studies to provide a more complete assessment of the effect of human presence on the system. Although there have been several studies on the water quality of coastal cities of the Galapagos and how it is related to population size and the number of incoming tourists, the pressures themselves (i.e. agricultural runoffs and untreated wastewater discharge) remain poorly understood. Mateus et al. (2019), for example, suggested that elevated phosphorus concentrations in the coastal waters of Santa Cruz could have been the result of inland pollution and water discharge, but studies to confirm this or to provide a more detailed description of the pollution sources and pathways remain absent \citep{Mateus2019AnArchipelago}.

This study can serve as baseline for future studies aiming to improve ecological understanding and/or develop environmental management guidelines of the Galapagos archipelago and other tropical oceanic islands that face similar threats of increasing anthropogenic pressures. Future studies should focus on the development of water quality criteria adapted to local conditions, identification of pressures on the environment and management prioritization of pressures and associated environmental changes with the strongest effect on local ecosystems. 



 


