\section{Discussion}
\label{Discussion}

Although the $\alpha$ diversity, $\beta$ diversity and structure of the fish assemblages were clearly different between both islands, elucidating the drivers of these differences is not evident because of the complex interplay of often intercorrelated environmental variables and spatially confounded factors (e.g. occurrences of natural upwelling and anthropogenic pollution) \citep{Legendre2012NumericalEcology,Robinson2011PushingOpportunities}. It is however advisable to consider such a large set of potential drivers as other metrics to characterize fish assemblages of similar islands in the Tropical Eastern Pacific have been found to be affected by multiple different environmental, biogeographical and anthropogenic factors \citep{Luiz2015CommunityVariables,Quimbayo2019DeterminantsIslands,Quimbayo2017UnusualPacific}. Nitrite, ammonium, sulfate, DO concentration, temperature and pH were significantly different between both islands and sand cover was significantly higher within the transects of Santa Cruz. In addition, variability of the water conditions and physical habitats were both larger in Santa Cruz than in Floreana. Furthermore, Santa Cruz was characterized by strong anthropogenic pressures and little natural upwelling, while the contrary was true for Floreana. Because the series of variables were characterized by many correlations among their variables, there was no sound statistical ground to separate the effects of geographical distance, water quality and physical habitats on diversity and the structure of fish assemblages based on this observation alone. However, by collecting data over multiple scales, i.e. islands, locations and transects, a more detailed assessment could be made \citep{Yackulic2016}.

\subsection{Diversity}

Although the point species richness and sample species richness did not seem to differ between the islands, the total number of species and $\beta$ diversity in Santa Cruz were higher than in Floreana. This suggests that on a habitat scale, i.e. within transects and within locations, the number of species may have been the same, but that the variability in species composition was larger in Santa Cruz than in Floreana. This is expected as the variability in observed water conditions and physical habitats of the different locations was also higher in Santa Cruz than in Floreana and more different niches are more likely to harbor more different species \citep{Elith2009,Guisan2000}. In addition, the $\beta$ diversity did not differ significantly between the locations, which corroborates the absence of significant differences in the variability of physical habitats between the locations. However, it should be noted that the number of transects per location, and therefore the statistical power, was limited. The fact that the $\beta$ diversity of locations showed a weak positive correlation with physical habitat variability and that $\beta$ diversity of the islands seemed positively affected by the variability of water conditions and physical habitats, suggests that environmental conditions may be important for diversity on multiple spatial scales. Nevertheless, more fine-scale studies are required.

In Santa Cruz, lower levels of $\alpha$ diversity (both point and sample species richness) were associated with higher nutrient concentrations, which is in line with the results of meta-analyses on the effects of anthropogenic pollution on marine communities \citep{Johnston2009ContaminantsMeta-analysis}. Nitrate and phosphate concentrations measured during this study in Santa Cruz were found to be four and two times higher than those recorded in June 2007 \citep{Palacios2009Estudio2007.} and eight and seven times higher than those recorded in 1968 \citep{Houvenaghel1978OceanographicIslands}, underlining the increasing risk fish communities are facing. In addition, Mateus et al. (2019) found some strong relationships between population increase and nutrient concentrations in coastal sites of Santa Cruz during a 9-year study period \citep{Mateus2019AnArchipelago}. However, whether the underlying cause behind elevated nutrient concentrations is natural or man-made remains often hard to determine. This is especially the case for the hydrodynamically complex area of the Galapagos archipelago, being subject to local upwelling of cold, saline, nutrient rich waters originating from the eastward Cromwell current \citep{Houvenaghel1978OceanographicIslands,Mateus2019AnArchipelago,Schaeffer2008PhytoplanktonMeasurements}. A combination of satellite imagery, remote sensing based models and in situ nutrient measurements indicated that Floreana was characterized by strong natural upwelling, while high nutrient concentrations in Santa Cruz were more likely the result of anthropogenic pollution (section S3.3). This result corroborates results of earlier studies \citep{Mateus2019AnArchipelago,Werdeman2006EffectsBays}. Furthermore, the fact that fecal coliforms and \textit{E. coli} were found more often in the coastal waters of Puerto Ayora, indicates a stronger anthropogenic pollution compared to Floreana \citep{Fernandez2008CoastalIsland,Stumpf2013InvestigatingEcuador}.

%observed in locations with higher nutrient concentrations, which are most likely the result of anthropogenic pressures.  The increased nutrient concentrations in combination with high water temperatures are likely to benefit primary production in Santa Cruz, explaining the relatively high chlorophyll concentrations. This increased primary production is likely to benefit competitively superior species, causing their disproportionate dominance over other species and negative effect on diversity \citep{Johnston2009ContaminantsMeta-analysis}. 
% as meta-analyses of studies on marine communities already suggested \citep{Johnston2009ContaminantsMeta-analysis}. 
% Although measured nutrient concentrations between the islands did not differ much, primary production is likely to be different. the low temperature in Floreana is likely to limit primary production (explaining the relatively low chlorophyll concentrations), the high temperature in combination  The high nutrient concentrations in combination with high temperature, might result in a higher primary productivity and a disproportionate presence of competitively superior species. 

\subsection{Structure of fish assemblages}

\subsubsection{Characteristic species}

DO and pH were significantly higher in Santa Cruz and seemed to have a positive effect on the abundance of the two most typical species from Santa Cruz, the Yellowtail Damselfish and the Pacific Spottedfin Mojarra. Hence, the absence of these species in Floreana may be related to the lower DO and pH. However, the abundances of the two most typical species of Floreana, the Galapagos Ringtail Damselfish and Chameleon Wrasse, were correlated only with nitrate and phosphate gradients in Floreana, respectively. These variables were not significantly different between the islands, hence providing no indication that environmental conditions are responsible for the relative absence of these species in Santa Cruz. However, it should be noted that the number of monitored locations per island was limited and that the measured environmental gradients in Floreana were less pronounced than in Santa Cruz.

\subsubsection{Fish assemblages}

The results suggest that the prevailing environmental conditions, rather than the geographical distance between the islands, are the underlying cause for the observed differences in the structure of the fish assemblages. First, in Santa Cruz and Floreana a mixture of Endemic, Peruvian, Indo-Pacific, Panamic and widespread species were found, confirming the conclusion of Edgar et al. (2004) regarding the strong connectivity of the archipelago with the surrounding islands and coasts \citep{Edgar2004}. Second, since within the islands of Santa Cruz and Floreana, temperature and the cover of sand turned out to be most important for the structure of fish assemblages and given the significant differences of these variables between both islands, it is likely that environmental conditions, rather than geographical distances, are the drivers behind the observed differences in the structure of the fish assemblages of both islands. In addition, the significantly stronger between-locations variability of the structure of the fish assemblages in Santa Cruz compared to Floreana, is likely the result of the pronounced environmental differences between the locations of Santa Cruz compared to those of Floreana. %The importance of environmental conditions for the structure of fish assemblages has also been established by studies on other islands in the Tropical Eastern Pacific \citep{ZapataF.A.Morales1997SpatialColombia,Quimbayo2019DeterminantsIslands}.

Although multiple studies on fish assemblage structures of tropical oceanic islands have highlighted the importance of the degree of isolation, there have been conflicting results and even indications that other, often confounding, factors are at least equally important \citep{Hobbs2011ExtinctionFishes,Luiz2015CommunityVariables,Quimbayo2019DeterminantsIslands}. Due to its unique geographical position at the intersection of multiple oceanic currents and the resulting wide range of environmental conditions, the Galapagos archipelago provides a unique case to assess the importance of these other factors. Indeed, despite its comparable degree of geographical isolation, the Galapagos has a relatively high species richness and functional dispersion compared to other tropical oceanic islands \citep{Giddens2019PatternsPacific,Quimbayo2019DeterminantsIslands}. The importance of the wide range of environmental conditions, in a relatively small area, for the fish assemblage structure had already been suggested in other studies \citep{Edgar2004,Stuart-Smith2013IntegratingDiversity}, but this is the first study to provide quantitative results in favor of this hypothesis. 

% has been found to be more likely the result of ts high surface area and unique position at the intersection of multiple oceanic currents

% Studies on the effect of the degree of isolation of oceanic islands on multiple structure metrices have provided conflicting results and have indicated that multiple other factors are often more important  \citep{Hobbs2011ExtinctionFishes,Luiz2015CommunityVariables,Quimbayo2019DeterminantsIslands}. Unlike most isolated oceanic islands, the Galapagos archipelago has a relatively high species richness and functional diversity. This is most likely the result of its high surface area and unique position at the intersection of multiple oceanic currents, providing a wide range of environmental conditions in a relatively small area . 

% Although, some studies have indicated the degree of isolation of oceanic islands as an important factor to negatively affect the structure of fish assemblages \citep{Hobbs2011ExtinctionFishes,Luiz2015CommunityVariables}, a more recent study by \cite{Quimbayo2019DeterminantsIslands} described a positive effect for multiple metrics describing the fish assemblage structure. 

%Compared to most other oceanic islands the Galapagos archipelago has a 

Fish assemblages in Santa Cruz, an island characterized by anthropogenic pollution \citep{Werdeman2006EffectsBays,Mateus2019AnArchipelago}, are affected more by water conditions than physical habitats. The contrary is true for Floreana, where anthropogenic pollution is limited, invoking smaller gradients in water conditions (especially for those variables that turned out to be potential predictors for the structure of fish assemblages). To predict the structure of fish assemblages, nutrient concentrations, i.e. Total P, and temperature gradients were most important in Santa Cruz, while in Floreana the percentage sand cover was most important. Although sand cover within the observed transects was relatively low in Floreana, significantly different fish assemblage structures were observed along transects with high versus low sand cover.

Although the most parsimonious model for both islands only contained the geographical distance, latitude and longitude provided clearly a better representation of the geographical distance than ammonium, temperature and all other measured variables could ever provide for the environmental state of the water. Despite being inherent to any spatial environmental study, this limitation can be partially accounted for by including multiple spatial scales in the sampling design, as was illustrated here. As such, the main drivers behind the observed regional and local differences in fish assemblages could be identified with some level of statistical confidence, adding to the work of Harris (1969), Jennings et al. (1994) and Edgar et al. (2004) \citep{Harris1969BreedingIslands,Jennings1994TheArchipelago,Edgar2004}. As suggested by these authors, temperature was identified as a major driver of fish assemblage structure. However, nutrient concentrations and the characteristics of the prevailing physical habitats play an additional important role. 