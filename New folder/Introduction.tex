%% main text
\section{Introduction}
\label{Introduction}

% Our ability to predict the response of biological communities to changes in the environment such as increased pollution levels, climate change, habitat degradation or the construction of movement barriers, strongly depends on our understanding of ecological and biological species traits \citep{Costello2015}. 
% Although fish are known to have preferences in terms of water conditions and physical habitats, species-environment relationships are often difficult to describe because of the different spatial scales over which different processes take place \citep{Robinson2011PushingOpportunities,Bruneel2018}. In addition, displacement within and between habitats may allow fish to be present in unsuitable habitats, while dispersal barriers and biotic interactions may hamper fish species from reaching and persisting in suitable habitats \citep{Heino2015,Meynard2013}. 

% Although water conditions such as temperature \citep{Mora2003PatternsDiversity}, salinity and nutrients \citep{Mellin2019EnvironmentalFishes} in combination with habitat structure have been identified as major drivers of the structure and diversity of fish assemblages \citep{Komyakova2013RelativeCommunities}, spatial factors should not be overlooked \citep{Araujo2006,Bruneel2018}. For example, reef fish often exhibit a strong site fidelity \citep{Chapman2000MovementsBarbados} and isolated islands tend to harbor fewer species \citep{Cowen2006ScalingPopulations,Whittaker2001ScaleDiversity}. In addition, dispersal ranges of the larvae of different fish species often do not exceed 10 to 100 km \citep{Cowen2006ScalingPopulations,Cowen2009LarvalConnectivity}. However, marine species distribution models typically do not consider dispersal limitation because of the presumed greater potential for dispersion in marine systems compared to terrestrial and freshwater systems \citep{Mouton2007,Robinson2011PushingOpportunities}. 
% In addition, because of oceanic currents it is often difficult to assess whether the structure of marine fish assemblages is more affected by dispersal limitation or by water conditions. This is because oceanic currents, which are vital for larvae dispersal and cause abrupt changes in environmental conditions, might act as either physical or physiological barriers respectively \citep{Gaylord2000TemperatureFlow}. For example, \cite{Sandin2008IslandFish} were not able to disentangle the effects of productivity, an environmental variable, and isolation, associated with dispersal limitation, on fish species richness due to their negative correlation. Therefore, studies that focus on a limited set of potential predictors without accounting for different confounding factors may yield questionable results. 

Due to their isolated position, oceanic islands harbor some of the last near-pristine marine ecosystems and associated fish assemblages \citep{Sandin2008IslandFish,Friedlander2016MarineHotspots}. Anthropogenic pressures such as fishery, pollution and the introduction of invasive species threaten however many of these already fragile marine communities and sound evidence-based management is required to protect them from extinction \citep{Vitousek1988ChapterIslands,Wilson2010HabitatCommunities}. Among these oceanic islands, the Galapagos archipelago is key due to its exceptionally rich biodiversity and its role as stepping stone between the Tropical Eastern and Central Pacific \citep{Edgar2004,Quimbayo2019DeterminantsIslands}. %To enable sound management of the ecosystems of the Galapagos, insights in how the structure and diversity of the fish assemblages are affected by environmental factors and anthropogenic pressures is required.  
Although the rich biodiversity of the Galapagos archipelago has been attributed to it being located at the intersection of multiple warm and cold ocean currents \citep{Palacios2004SeasonalInfluences}, it remains unclear what the drivers behind the diversity and structure of the local fish assemblages are. 

The observed differences in the marine communities seem to coincide with sharp differences in environmental conditions, which have given rise to the delineation of multiple bio-geographical regions \citep{Harris1969BreedingIslands,Jennings1994TheArchipelago}. The strong regional divisions in fish assemblages observed by Edgar et al. (2004) were considered to reflect both the local environmental conditions and connectivity of fish larvae with external source regions, such as the Indo-Pacific, Panamic and Peruvian region \cite{Edgar2004}. However, at the same time, the high species richness on the far-northern isolated islands of Wolf and Darwin suggested high immigration rates and a strong connectivity of fish assemblages between the islands \citep{Edgar2004}. In the case of high inter-island connectivity, it is expected that fish are able to reach the different available habitats and that their prevalence is mainly determined by local environmental conditions, rather than by dispersal limitation. Based on differences in temperature, Harris (1969) identified five potential bio-geographical regions in the archipelago \cite{Harris1969BreedingIslands}, but these were significantly different from the regions proposed by Edgar et al. (2004), which were based on differences in the structure of fish and macroinvertebrate assemblages \cite{Edgar2004}. According to Edgar et al. (2004), there was no sound evidence to subdivide the area east of Isabela and south of Marchena in the three zones suggested by Harris (1969) \cite{Edgar2004,Harris1969BreedingIslands}. Since the within-island biological variability of this Central-Southeastern zone was larger than the between-island biological variability \citep{Edgar2004}, local differences in environmental conditions, e.g. temperature and physical habitat characteristics, may indeed have been crucial to shape the observed fish assemblages. Nevertheless, studies that assess the effect of local environmental conditions have been few. Jennings et al. (1994) and Edgar et al. (2004) provided strong evidence for differences in fish assemblages, but they were unable to identify any relationships with environmental conditions due to lacking data \cite{Edgar2004,Jennings1994TheArchipelago}. Llerena-Martillo et al. (2018) combined visual census data with environmental measurements to study fish assemblages in mangrove ecosystems in Santa Cruz, but the number of sites and number of variables remained limited \cite{Llerena-Martillo2018FishReserve}.

Besides natural stressors, anthropogenic pressures may also affect fish assemblages. During the last three decades, the Galapagos archipelago has been witnessing an increase in population, tourism and waste production at an annual rate of 4.08, 6.71 and 4.02 \% respectively, but good waste management procedures have not been evolving at the same pace \citep{Alava2014PollutionPerspectives,Fernandez2008CoastalIsland,Hardter2010WasteIslands,MinistryofTourismofEcuador2019Observatorio2019-09-12,Riascos-Flores2020PollutedEcuador}. Since pollution has been identified as a major threat to marine ecosystems \citep{Boersma1999LimitingSolution}, understanding the drivers behind the structure of fish assemblages in the Galapagos archipelago is crucial to evaluate and/or propose conservation guidelines \citep{Mateus2019AnArchipelago}.  

The aim of this study was to identify the factors responsible for patterns in diversity and the structure of fish assemblages on the rocky shores of the Central-Southeastern zone of the Galapagos archipelago. To this end, the coastal areas of two cities, significantly different in size, on two different islands were assessed using underwater video transects. 
Difficulties in separating the effects of the potential drivers, i.e. dispersal limitation, water conditions and physical habitats, were partially circumvented by applying a multi-scale approach \citep{Thuiller2015}. First, dispersal limitation can prevent fish to track and respond to environmental differences and therefore can affect ecologically processes at a larger, more regional scale (>100 kilometer) than environmental preferences \citep{Bruneel2018,Leibold2004,Leibold2017}. Second, the wide range of water conditions in the archipelago, which are the result of distinct oceanic currents and urbanization, are more likely to affect communities at a more intermediate scale (0.1-100 kilometer) \citep{Houvenaghel1978OceanographicIslands,Liu2014,Mateus2019AnArchipelago}. Finally, the composition of physical habitats (e.g. sand and bare rock) seems to mainly vary at a relatively fine scale \citep{Okey2004AStrategies}, causing it to potentially affect communities at a more local scale (1-100 meter).  
A hierarchical multi-scale repeated-observations sampling design was used to collect biological data in the Central-Southeastern zone of the archipelago. These data were used to compare fish assemblages between islands, between locations and within locations. 

Given the expected differences in variation and magnitude of anthropogenic pressures and environmental conditions between and within the islands of Santa Cruz and Floreana, we hypothesize that fish assemblage structure and diversity will differ accordingly and will be affected by different factors. We hypothesize that in the highly populated bay of Santa Cruz, local anthropogenic pressures induce pronounced gradients in water conditions, which affect the structure and diversity of the fish assemblages. Fish diversity is expected to be negatively affected by human-induced changes of the water quality. In the sparsely populated bay of Floreana, we expect less pronounced gradients in water conditions and therefore a stronger effect of the composition of the physical habitat.    

Although the main aim of this study was to determine the main factors that steer the fish assemblages of the Galapagos, the obtained results are also of direct use for the many other tropical oceanic islands that face increasing anthropogenic pressures.   

% In Santa Cruz, local anthropogenic pressures are expected to cause a more pronounced difference in water conditions between 
% We hypothesize that the presence of locations with stronger anthropogenic pressures in Santa Cruz, 
% there is a pronounced gradient of anthropogenic pressures in Santa Cruz,
% differences in fish assemblage structure within the highly populated island of Santa Cruz are more likely the result of water conditions   